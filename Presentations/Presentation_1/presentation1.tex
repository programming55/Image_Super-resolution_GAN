\documentclass{beamer}
\usepackage{hyperref}
\hypersetup{
	colorlinks,
	citecolor=white,
	filecolor=white,
	linkcolor=white,
	urlcolor= cyan
}

\usetheme[progressbar=foot, background=dark]{metropolis}  
\setbeamertemplate{frame footer}{Ankit Pant - 2018201035}

\title{\centering Introduction to \\ Image Super Resolution \\ and \\ Generative Adversarial Networks \\}
\author{Ankit Pant \\ 2018201035}
\date{}
\begin{document}
\begin{frame}[plain]
    \maketitle
\end{frame}
\begin{frame}{Outline}
	\setbeamertemplate{section in toc}[sections numbered]
	\tableofcontents
\end{frame}

\section{Introduction}
	\begin{frame}{Introduction}
		\begin{itemize}
			\item Artificial Intelligence (AI) has boomed tremendously during the last few years
			\item Various AI models have been developed (e.g. GANs, Deep Networks, etc.)
			\item Generative Adversarial Model - among various models used in AI and Machine Learning (ML)
			\item AI today is being used for a large number of applications
			\item  Image Super-resolution is one such application
		\end{itemize}
	\end{frame}

\section{Image Super-Resolution}
	\begin{frame}{Basics \& Terminology }
		\begin{itemize}
			\item 	\textbf{Image Super-resolution:} Conversion of (one or more ) low resolution images into a high resolution image 
			\item Benefits of increasing image resolution:
			\begin{itemize}
				\item The resultant image is larger
				\item It provides more details
				\item Can be used to improve image quality as well as video quality
			\end{itemize}
		\end{itemize}
	\end{frame}
	
	\begin{frame}{Increasing Image Resolution}
		\begin{itemize}
			\item \textbf{Reducing Pixel size:}
				\begin{itemize}
					\item Increases the number of pixels per unit area
					\item Advantage: Increases spatial resolution
					\item Disadvantage: Introduction of Noise
				\end{itemize}
			\item \textbf{Increase Chip Size:}
				\begin{itemize}
					\item Hardware based solution
					\item Advantage: Enhances spatial resolution
					\item Disadvantage: Expensive
				\end{itemize}
			\item \textbf{Image Super-resolution:}
				\begin{itemize}
					\item Combines multiple low resolution images to form high resolution image
					\item Advantage: Less expensive
					\item Disadvantage: May not be always accurate
				\end{itemize}
		\end{itemize}
	\end{frame}
	
	\begin{frame}{Image Super-resolution - Technique}
		\begin{itemize}
			\item \textbf{Single-frame Super-resolution}
			\begin{itemize}
				\item Traditional resolution enhancement  - includes smoothing, interpolation and sharpening
				\item Estimates detail that is not present
				\item Training-set used to learn details of images at low resolution
				\item These learned relationships used to predict details of other images
			\end{itemize}
			\item \textbf{Multi-frame Super-resolution}
			\begin{itemize}
				\item Works if multiple low resolution images are available of the same scene
				\item Each image is naturally shifted with sub-pixel precision
				\item Works when each of the images have different sub-pixel shifts
			\end{itemize}
		\end{itemize}
	\end{frame}

	\begin{frame}{Applications of Image Super-resolution}
		\begin{itemize}
			\item Enhancing surveillance footage
			\item Enhancing medical diagnostic images
			\item Enhancing astronomical and remotes sensing images
			\item Enhancing low resolution videos (from the past)
			\item Enhancing photographs and self-portraits of people
		\end{itemize}
	\end{frame}


\section{Generative Adversarial Networks}
	\begin{frame}{Basics \& Terminology}
		\begin{itemize}
			\item Generative Adversarial Networks (GANs) are AI models that contains a combination of two  models - generator model and discriminator model
			\item \textbf{Generator model:}
			\begin{itemize}
				\item It is responsible for generating images (usually from noise)
			\end{itemize}
			\item \textbf{Discriminator model:}
			\begin{itemize}
				\item It is responsible to determine whether the image was originally available or generated by the generator 
			\end{itemize}
			\item Both the models work in tandem:
			\begin{itemize}
				\item Generator tries to recreate the image as faithfully as possible so that the discriminator cannot differentiate between original image and generated image
				\item Discriminator tries to identify the generated images from the original images 
			\end{itemize}
		\end{itemize}
	\end{frame}

	\begin{frame}{Sample Training Phrase of a GAN}
		\begin{itemize}
			\item The generator tries to recreate original image from noise
			\item This image is then input to discriminator which tries to identify whether it was generated
			\item The generator then again improves on the generated images
			\item The discriminator again determines whether the images was generated or not
			\item This process is repeated until the discriminator can no longer determine whether the image was generated or not $P(generated) = P(original) = 0.50$
			\item Both the generator and discriminator may be pre-trained to improve performance
		\end{itemize}
	\end{frame}

	\begin{frame}{Types of GANs}
		\begin{itemize}
			\item \textbf{DCGAN}
			\begin{itemize}
				\item Deep Convolutional GANs
				\item Consists of convolution layers without max pooling or fully connected layers
				\item Use transposed convolution for upsampling
			\end{itemize}
			\item \textbf{WGAN}
			\begin{itemize}
				\item Wasserstein GAN
				\item Attempts to solve vanishing gradient problem of regular GANs
				\item WGAN learns no matter the generator is performing or not
			\end{itemize}
		\item \textbf{Softmax GAN}
		\begin{itemize}
			\item  Replaces the classification loss (regular GAN) with a softmax cross-entropy loss 
			\item  Stabilizes GAN training
		\end{itemize}
		\end{itemize}
	\end{frame}

\section{Image Super-resolution using GAN}
	\begin{frame}{Image Super-resolution using GAN}
		\begin{itemize}
			\item Single-frame Super-resolution will be attempted using the following procedure:
			\begin{itemize}
				\item High resolution images will be converted to low resolution images for training
				\item Generator model will be trained on training set individually
				\item Discriminator model will be trained on training set individually
				\item Both the models will be combined in a GAN and trained
				\item Test data-set consisting of low resolution images will be used to gauge  performance of the model
			\end{itemize}
			\item After the model has been trained and saved, a simple web-application will be developed to convert low resolution images input by users to high resolution images
		\end{itemize}
	\end{frame}

\section{References}
	\begin{frame}{References}
		\begin{thebibliography}{10}
			\bibitem{1} Introduction to Image Super-resolution, Kevin Su, \url{http://www.cs.utsa.edu/~qitian/seminar/Fall04/superresolution/SR\_slides\_xsu.pdf}
			\bibitem{2} Generative adversarial network, Wikipedia, the free encyclopedia, \url{https://en.wikipedia.org/wiki/Generative_adversarial_network}
			\bibitem{3} Generative Adversarial Nets, Ian J. Goodfellow et al.			
			\bibitem{4} GAN-DCGAN, Jonathan Hui, \href{https://medium.com/@jonathan_hui/gan-dcgan-deep-convolutional-generative-adversarial-networks-df855c438f}{medium.com/gan-dcgan}
			\bibitem{5}GAN — Wasserstein GAN, \url{https://mc.ai/gan-wasserstein-gan-wgan-gp/}
			\bibitem{6} Softmax GAN, Min Lin
		\end{thebibliography}
	\end{frame}

\end{document}
